%\documentclass[12pt,a4paper]{scrartcl}
\documentclass[12pt,a4paper]{article}

\makeatletter % Technical doc - START

\usepackage[utf8]{inputenc}
\usepackage[T1]{fontenc}

\usepackage{dsfont}

\usepackage[french]{babel,varioref}

\usepackage[top=2cm, bottom=2cm, left=1.5cm, right=1.5cm]{geometry}
\usepackage{enumitem}

\usepackage{color}
\usepackage{hyperref}
\hypersetup{
    colorlinks,
    citecolor=black,
    filecolor=black,
    linkcolor=black,
    urlcolor=black
}

\usepackage{import}

\usepackage{multicol}

\usepackage{tcolorbox}

\usepackage{amsthm}

\usepackage{ifplatform}

\usepackage{fancyvrb}

\usepackage{cbdevtool}
\usepackage{lymath}

% MISC

\setlength{\parindent}{0cm}
\setlist{noitemsep}


\theoremstyle{definition}
\newtheorem*{remark}{Remarque}


\usepackage[raggedright]{titlesec}

\titleformat{\paragraph}[hang]{\normalfont\normalsize\bfseries}{\theparagraph}{1em}{}
\titlespacing*{\paragraph}{0pt}{3.25ex plus 1ex minus .2ex}{0.5em}

\setlength{\parindent}{0cm}

\newenvironment{frame-gene}[1][]{
	\begin{tcolorbox}[
		title        = #1, 
		colbacktitle = black!10!white, 
		colback      = white, 
		coltitle     = black,
		fonttitle    = \bfseries\itshape\small, 
		breakable,
		center title]
}{
	\end{tcolorbox}
}

\newcommand\myquote[1]{{\itshape \og #1 \fg}}

\newcommand\squaremacro{$x^2$}


\newenvironment{latex-frame}{
	\begin{frame-gene}[Code \LaTeX{}]
}{
	\end{frame-gene}
}


\newcommand\justcode[1]{
	\begin{latex-frame}
		\small
		\VerbatimInput{#1}
	\end{latex-frame}
}


\newcommand\codeasideoutput[1]{
    \begin{multicols}{2}
    	\centering
    	\justcode{#1}
    	\vfill\null
    	\columnbreak
    	\textit{Mise en forme correspondante.}

    	\vspace{1em}
    
 	    \small
    	\input{#1}
	    \vfill\null
    \end{multicols}
 }

\makeatother % Technical doc - END


\usepackage{lyalgo}


\begin{document}

\renewcommand\labelitemi{\raisebox{0.125em}{\tiny\textbullet}}
\renewcommand{\labelitemii}{---}

\title{%
	Le package \texttt{lyalgo}:
	\\
	taper facilement de jolis algorithmes
	\\
	{
		\footnotesize Code source disponible
		sur \url{https://github.com/bc-latex/ly-algo}.%
	}
	\\
	{
		\footnotesize Version \texttt{0.1.1-beta}
		développée et testée sur \macosxname{}.%
	}
}

\author{Christophe BAL}
\date{2019-10-23}

\maketitle


\vspace{2em}

\hrule

\tableofcontents

\vspace{1.5em}

\hrule

\newpage



\section{Introduction}

Le but de ce package est d'avoir facilement des algorithmes
\footnote{
	Le gros du travail est fait par \texttt{algorithm2e}.
}
ainsi que des contenus \verb+verbatim+ un peu flexibles
\footnote{
	Tout, ou presque, est géré par \texttt{alltt}.
}.
Les algorithmes mis en forme ne sont pas des flottants, par choix, et ils utilisent une mise en forme proche de la syntaxe \verb+Python+.




\section{\texttt{lymath}, un package qui vous veut du bien}

Le package \verb+lymath+ est un bon complément à \verb+lyalgo+ : voir à l'adresse \url{https://github.com/bc-latex/ly-math}. 
Il est utilisé par cette documentation pour simplifier la saisie des formules.




\section{Écriture \texttt{pseudo-verbatim}}

En complément à l'environnement \verb+verbatim+ est proposé l'environnement \verb+pseudoverb+, pour \og pseudo verbatim \fg, qui permet d'écrire du contenu presque verbatim : ci-après, la macro \verb+\squaremacro+ définie par \verb+\newcommand\squaremacro{$x^2$}+ est interprétée mais pas la formule mathématique. 


\codeasideoutput{examples/pseudo-verb/with-remark.tex}


\vspace{-1em}

Il est en fait plus pratique de pouvoir taper quelque chose comme ci-dessous avec un cadre autour où le titre est un argument obligatoire \emph{(voir plus bas comment ne pas avoir de titre)}.


\codeasideoutput{examples/pseudo-verb/console.tex}


\vspace{-1em}

Finissons avec une version bien moins large et sans titre de la sortie console ci-dessus. Le principe est de donner un titre vide via \verb+{}+, \textbf{c'est obligatoire}, et en utilisant l'unique argument optionnel pour indiquer la largeur relativement à celle des lignes \emph{(attention aux environnements multi-colonne qui ont des lignes moins larges que celles du corps principal du document)}.


\codeasideoutput{examples/pseudo-verb/scaled.tex}


\vspace{-1em}

\begin{frame-gene}[À RETENIR]
	\centering\itshape
	C'est la version étoilée de \verb+pseudoverb+ qui en fait le moins.
	
	Ce principe sera aussi suivi pour les algorithmes. 
\end{frame-gene}




\newpage

\section{Algorithmes en langage naturel}

\subsection{Comment taper les algorithmes}

Le package \verb+algorithm2e+ permet de taper des algorithmes avec une syntaxe simple. La mise en forme par défaut de \verb+algorithm2e+ utilise des flottants, chose qui peut poser des problèmes pour de longs algorithmes ou, plus gênant, pour des algorithmes en bas de page. Dans \verb+lyalgo+, il a été fait le choix de ne pas utiliser de flottants, un choix lié à l'utilisation faite de \verb+lyalgo+ par l'auteur pour rédiger des cours de niveau lycée. 


\medskip

Dans la section suivante, nous verrons comment encadrer les algorithmes. Pour l'instant, voyons juste comment taper l'algorithme suivant où tous les mots clés sont en français. Indiquons au passage l'affichage du titre de l'algorithme en haut et non en bas comme cela est proposé par défaut.


\bigskip
\begin{algo*}
    \caption{Suite de Collatz $(u_k)$ -- Conjecture de Syracuse}

    \Data{$n \in \NN$}
    \Result{le premier indice $i \in \ZintervalC{0}{10^5}$ tel que $u_i = 1$ 
            ou $(-1)$ en cas d'échec}

    \addalgoblank    % Pour aérer un peu la mise en forme.

    \Actions{
        $i, imax \Store 0, 10^5$
        \\
        $u \Store n$
        \\
        $continuer \Store \top$
        \\
        \While{$continuer = \top \And i \leq imax$}{
            \uIf{$u = 1$}{
                \Comment{C'est gagné !}
                $continuer \Store \bot$
            } \Else {
                \Comment{Calcul du terme suivant}
                \uIf{$u \equiv 0 \,\, [2]$}{
                    $u \Store u / 2$
                    \Comment*{Quotient de la division euclidienne.}
                } \Else {
                    $u \Store 3u + 1$
                }
                $i \Store i + 1$
            }
        }
        \If{$i > imax$}{
            $i \Store (-1)$
        }
        \Return{$i$}
    }
\end{algo*}

\bigskip


La rédaction d'un tel algorithme est facile car il suffit de taper le code suivant proche de ce que pourrait proposer un langage classique de programmation. Le code utilise certaines des macros additionnelles proposées par \verb+lyalgo+ \emph{(voir la section \ref{algo-extra})} ainsi que la macro \verb+\ZintervalC+ du package \verb+lymath+. Nous donnons juste après le squelette de la syntaxe propre à \verb+algorithm2e+.


\justcode{examples/algo-basic/syracuse.tex}


Le squelette du code précédent est le suivant. 


\begin{frame-gene}[Squelette du code \texttt{algorithm2e}]
    \small
\begin{verbatim}
\caption{...}

\Data{...}
\Result{...}

\Actions{
    ...
    \While{...}{
        \uIf{...}{
            ...
        } \Else {
            ...
            \uIf{...}{
                ...
            } \Else {
                ...
            }
            ...
        }
    }
    \If{...}{
        ...
    }
    \Return{...}
}
\end{verbatim}
\end{frame-gene}




%\section{Algorithmes en langage naturel}

\subsection{Numérotation des algorithmes}


Avant de continuer les présentations, il faut savoir que les algorithmes sont numérotés globalement à l'ensemble du document. C'est plus simple et efficace pour une lecture sur papier.




%\section{Algorithmes en language naturel}

\subsection{Des algorithmes encadrés}

La version non étoilée de l'environnement \verb+algo+ ajoute un cadre, comme ci-dessous, afin de rendre plus visibles les algorithmes. Indiquons au passage que nous avons utilisé la macro \verb+\dsum+ fournie par le package \verb+lymath+.

\codeasideoutput{examples/algo-basic/stupid.tex}


\vspace{-1em}


L'environnement \verb+algo+ propose un argument optionnel pour indiquer la largeur relativement à celle des lignes.
Ainsi  via \verb+\begin{algo}[.45] ... \end{algo}+, on obtient la version suivante dont le cadre ne couvre pas toute la largeur de la ligne.

\begin{algo}[.45]
    \caption{Un truc bidon}

    \Data{$n \in \NNs$}
    \addalgoblank
    \Result{$\dsum_{i = 1}^{n} i$}

    \Actions{
        $s \Store 0$
        \\
        \ForRange{$i$}{$1$}{$n$}{
            $s \Store s + i$
        }
        \Return{$s$}
    }
\end{algo} 



\medskip


On peut utiliser un environnement \verb+multicols+ pour un effet sympa.

\begin{multicols}{2}
\begin{algo}
    \caption{Un truc bidon}

    \Data{$n \in \NNs$}
    \addalgoblank
    \Result{$\dsum_{i = 1}^{n} i$}

    \Actions{
        $s \Store 0$
        \\
        \ForRange{$i$}{$1$}{$n$}{
            $s \Store s + i$
        }
        \Return{$s$}
    }
\end{algo} 


\begin{algo}
    \caption{Un truc bidon}

    \Data{$n \in \NNs$}
    \addalgoblank
    \Result{$\dsum_{i = 1}^{n} i$}

    \Actions{
        $s \Store 0$
        \\
        \ForRange{$i$}{$1$}{$n$}{
            $s \Store s + i$
        }
        \Return{$s$}
    }
\end{algo} 
\end{multicols}




%\section{Algorithmes en langage naturel}

\subsection{Un titre minimaliste}

Dans l'exemple ci-dessous on voit un problème à gauche où l'on a utilisé \verb+\caption{}+ avec un argument vide pour la macro \verb+\caption+.
Si vous avec besoin juste de numéroter votre algorithme comme ci-dessous à droite, utiliser à la place \verb+\algovoidcaption+.


\begin{multicols}{2}    
\begin{algo}
    \caption{}

    \Datas{Un titre vide ci-dessus !}
    \Result{\dots}
    \Actions{
        \dots\phantom{X}
        \\
        \phantom{\dots}\vspace{-.75em}
    }
\end{algo}


\begin{algo}
    \algovoidcaption

    \Datas{\dots}
    \Result{\dots}
    \Actions{
        \dots\phantom{X}
        \\
        \phantom{\dots}\vspace{-.75em}
    }
\end{algo}
\end{multicols}




%\section{Algorithmes en langage naturel}

\subsection{Un premier ensemble de macros additionnelles ou francisées}

\begin{frame-gene}[À SAVOIR -- Le préfixe \texttt{u}]
	\centering\itshape
	Certains macros peuvent être préfixées par un \verb+u+ pour \myquote{unclosed} qui signifie \myquote{non fermé}.
	
	Ceci sert à ne pas refermer un bloc via un trait horizontal.
\end{frame-gene}


% ---------------- %


% == Block and words tools - START == %

\subsubsection{Entrée / Sortie}

Nous donnons ci-dessous les versions au singulier de tous les mots disponibles de type \myquote{entrée / sortie}.
Excepté pour \verb+\InState+ et \verb+\OutState+, toutes les autres macros ont une version pour le pluriel obtenu en rajoutant un \verb+s+ à la fin du nom de la macro.
Par exemple, le pluriel de \verb+\In+ s'obtient via \verb+\Ins+.


\codeasideoutput{examples/algo-basic/additional-macros/in-out.tex}
\codeasideoutput{examples/algo-basic/additional-macros/data-result.tex}
\codeasideoutput{examples/algo-basic/additional-macros/instate-outstate.tex}
\codeasideoutput{examples/algo-basic/additional-macros/precond-postcond.tex}

\subsubsection{Bloc principal}

Voici comment indiquer le bloc principal d'instructions avec deux textes au choix pour le moment.


\codeasideoutput{examples/algo-basic/additional-macros/main-block.tex}

\subsubsection{Boucles \TTfor{} et \TTwhile{}}

Voici les boucles de type \TTfor{} et \TTwhile{} proposées par le package.

\newpage


\codeasideoutput{examples/algo-basic/additional-macros/for-loop.tex}
\codeasideoutput{examples/algo-basic/additional-macros/forall-loop.tex}
\codeasideoutput{examples/algo-basic/additional-macros/foreach-loop.tex}
\codeasideoutput{examples/algo-basic/additional-macros/while-loop.tex}

\subsubsection{Boucles \TTrepeat{}}

Voici comment rédiger une boucle du type \TTrepeat{}.


\codeasideoutput{examples/algo-basic/additional-macros/repeat-loop.tex}

\subsubsection{Disjonction conditionnelle \TTif{}}

Les blocs conditionnels \TTif{} se rédigent très naturellement.


\codeasideoutput{examples/algo-basic/additional-macros/ifelif.tex}

\subsubsection{Disjonction de cas \TTswitch{}}

La syntaxe pour les blocs conditionnels du type \TTswitch{} ne pose pas de difficulté de rédaction.


\codeasideoutput{examples/algo-basic/additional-macros/switch.tex}

\subsubsection{Diverses commandes}

Pour finir voici un ensemble de mots supplémentaires qui pourront vous rendre service. Le préfixe \verb+m+ permet d'utiliser des versions \myquote{masculinisées} des textes proposés.

\newpage


\codeasideoutput{examples/algo-basic/additional-macros/word.tex}

% == Block and words tools - END == %


% ---------------- %


\subsection{Citer les outils de base en algorithmique}

Pour faciliter la rédaction de textes sur les algorithmes, des macros standardisent l'impression des noms des outils classiques de contrôle.
Dans les exemples qui suivent, les préfixes \verb+TT+ et \verb+AL+ font référence à \myquote{True Type} pour une police à chasse fixe, et à \myquote{AL-gorithme} pour une écriture similaire à celle utilisée dans les algorithmes.


% == Text tools - START == %

\begin{center}
	Liste des commandes de type \myquote{True Type}.
\end{center}

\begin{enumerate}
    \item \verb+\TTif+ donne \TTif.
    \item \verb+\TTfor+ donne \TTfor.
    \item \verb+\TTwhile+ donne \TTwhile.
    \item \verb+\TTrepeat+ donne \TTrepeat.
    \item \verb+\TTswitch+ donne \TTswitch.
\end{enumerate}

\begin{center}
	Liste des commandes de type \myquote{algorithme}.
\end{center}

\begin{enumerate}
    \item \verb+\ALif+ donne \ALif.
    \item \verb+\ALfor+ donne \ALfor.
    \item \verb+\ALwhile+ donne \ALwhile.
    \item \verb+\ALrepeat+ donne \ALrepeat.
    \item \verb+\ALswitch+ donne \ALswitch.
\end{enumerate}

% == Text tools - END == %




\newpage
\section{Des outils additionnels pour les algorithmes} \label{algo-extra}

\subsection{Convention en bosses de chameau}

Le package \verb+algorithm2e+ utilise, et abuse
\footnote{
	Ce type de convention est un peu pénible à l'usage.
},
de la notation en bosses de chameau comme par exemple avec \verb+\uIf+ et \verb+\Return+ au lieu de \verb+\uif+ et \verb+\return+.
Par souci de cohérence, les nouvelles macros ajoutées par \verb+lyalgo+ en lien avec les algorithmes utilisent aussi cette convention même si l'auteur aurait préféré proposer \verb+\putin+ et \verb+\forrange+ à la place de \verb+\PutIn+ et \verb+\ForRange+ par exemple.




%\section{Des outils pour les algorithmes} \label{algo-extra}

\subsection{Affectations simples ou multiples}

\subsubsection{Affectation simple avec une flèche}

Les affectations simples classiques $x \Store 3$ et $3 \PutIn x$ se tapent \verb+$x \Store 3$+ et \verb+$3 \PutIn x$+ respectivement où les macros \verb+\Store+ et \verb+\PutIn+ sont des opérateurs mathématiques.


\subsubsection{Affectation simple avec un signe égal décoré}

On peut aussi préférer la notation $x \Store* 3$ ou $x \Store** 3$. Ceci se tape via \verb+$x \Store* 3$+ et \verb+$x \Store** 3$+ respectivement.


\subsubsection{Affectation multiple}

Pour finir, les macros \verb+\MStore+ et \verb+\MPutIn+ servent pour les affectations multiples en parallèle comme dans $a, b, c \MStore x, y, z$ ou $x, y, z \MPutIn a, b, c$ pour indiquer que les dernières valeurs de $x$, $y$ et $z$ sont affectées aux variables $a$, $b$ et $c$.

\medskip

{\itshape \textbf{ATTENTION !} La multi-affectation se faisant en parallèle, le résultat de $a, b, c \MStore 2, a + b, c - b$ ne sera pas semblable à celui de $a \Store 2$ suivi de $b \Store a + b$ puis de $c \Store c - b$ car dans le second cas les variables $a$ et $b$ évoluent avant de nouvelles affectations simples. En fait la multi-affectation précédente correspond aux actions suivantes.}

\medskip

\begin{algo}
	\caption{Comment $a, b, c \MStore 2, a + b, c - b$ fonctionne-t-il ?}

	$a_{memo} \Store a$ ;
	$b_{memo} \Store b$ ;
	$c_{memo} \Store c$
	\\
	\addalgoblank
	$a \Store 2$
	\\
	$b \Store a_{memo} + b_{memo}$
	\\
	$c \Store c_{memo} - b_{memo}$
\end{algo}





%\section{Des outils pour les algorithmes} \label{algo-extra}

\subsection{Intervalles discrets d'entiers}

Il est d'usage en informatique théorique de poser $\CSinterval{4}{7} = \{ 4 ; 5 ; 6 ; 7 \}$ où \CSinterval{4}{7} s'écrit en tapant \verb+\CSinterval{4}{7}+. La syntaxe fait référence à \og Computer Science \fg{} soit \og Informatique Théorique \fg{} en anglais.




%\section{Des outils pour les algorithmes} \label{algo-extra}

\subsection{Listes}

\begin{frame-gene}[AVERTISSEMENT -- Premier indice]
	\centering\itshape
	Pour le package, les indices des listes commencent toujours à un. 
\end{frame-gene}


\subsubsection{Opérations de base.}

Voici les premières macros pour travailler avec des listes c'est à dire des tableaux de taille modifiable.

\begin{enumerate}
	\item \textit{Liste vide.}

		  \verb+\EmptyList+ imprime une liste vide $\EmptyList$.


	\item \textit{Liste en extension.}

		  \verb+\List{4 ; 7 ; 7 ; -1}+ produit $\List{4 ; 7 ; 7 ; -1}$.


	\item \textit{Le $k$\ieme{} élément d'une liste.}

		  \verb+\ListElt{L}{1}+ produit $\ListElt{L}{1}$.


	\item \textit{La sous-liste des éléments jusqu'à celui à la position $k$.}

		  \verb+\ListUntil{L}{2}+ produit $\ListUntil{L}{2}$.


	\item \textit{La sous-liste des éléments à partir de celui à la position $k$.}

		  \verb+\ListFrom{L}{2}+ produit $\ListFrom{L}{2}$.


	\item \textit{Concaténer deux listes.}

		  \verb+\AddList+ est l'opérateur binaire $\AddList$ qui permet d'indiquer la concaténation de deux listes.


	\item \textit{Taille ou longueur d'une liste.}

		  La macro \verb+\Len(L)+ produit $\Len(L)$.
\end{enumerate}



\subsubsection{Modifier une liste -- Versions textuelles}

\begin{enumerate}
	\item \textit{Ajout d'un nouvel élément à droite.}

	      \verb+\Append{L}{5}+ produit \myquote{\Append{L}{5}}
	      \footnote{
		       Le verbe anglais \myquote{append} signifie \myquote{ajouter}.
		  }.


	\item \textit{Ajout d'un nouvel élément à gauche.}

	      \verb+\Prepend{L}{5}+ produit \myquote{\Prepend{L}{5}}
	      \footnote{
		       Le verbe anglais \myquote{prepend} signifie \myquote{préfixer}.
		  }.


	\item \textit{Extraction d'un élément.}

	      \verb+\PopAt{L}{3}+ produit \myquote{\PopAt{L}{3}}.
\end{enumerate}



\subsubsection{Modifier une liste -- Versions POO}

Les versions étoilées des macros précédentes fournissent une autre mise en forme à la fois concise et aisée à comprendre
\footnote{
	L'opérateur point \POOpoint{} est défini dans la macro \texttt{\textbackslash{}POOpoint}. 
	Ceci permet de personnaliser facilement cet opérateur.
}
avec une syntaxe de type POO
\footnote{
	\myquote{POO} est l'acronyme de \myquote{Programmation Orientée Objet}.
}.


\begin{enumerate}
	\item \textit{Ajout d'un nouvel élément à droite.}

	      \verb+\Append*{L}{5}+ fournit \Append*{L}{5}.


	\item \textit{Ajout d'un nouvel élément à gauche.}

	      \verb+\Prepend*{L}{5}+ fournit \Prepend*{L}{5}.


	\item \textit{Extraction d'un élément.}

	      \verb+\PopAt*{L}{3}+ fournit \PopAt*{L}{3}.
\end{enumerate}



\subsubsection{Modifier une liste -- Versions symboliques}

Des versions doublement étoilées permettent d'obtenir des notations symboliques qui sont très efficaces lorsque l'on rédige les algorithmes à la main
\footnote{
	L'opérateur $\AddList$ est défini dans la macro \texttt{\textbackslash{}AddList}.
}.

\begin{enumerate}
	\item \textit{Ajout d'un nouvel élément à droite.}

	      \verb+\Append**{L}{5}+ donne \Append**{L}{5}.


	\item \textit{Ajout d'un nouvel élément à gauche.}

	      \verb+\Prepend**{L}{5}+ donne \Prepend**{L}{5}.


	\item \textit{Extraction d'un élément -- Version pseudo-automatique.}

	      \verb+\PopAt**{e}{L}{3}+ donne \PopAt**{e}{L}{3} avec un calcul fait automatiquement par la macro. Notez qu'ici on doit indiquer où stoker l'élément extrait.
	      
	      Bien entendu \verb+\PopAt**{e}{L}{1}+ produit \PopAt**{e}{L}{1} sans écrire $\ListUntil{L}{0} \AddList \ListFrom{L}{2}$ puisque pour le package les indices des listes commencent toujours à $1$.
	      
	      Il est autorisé de taper \verb+\PopAt**{e}{L}{k}+ pour obtenir \PopAt**{e}{L}{k}.
	      Par contre, \verb+\PopAt**{e}{L}{k-1}+ aboutit à \PopAt**{e}{L}{k-1} ce qui est très moche ! 
	      Dans ce cas, tapez \verb+\PopAt**{e}{L}{k-2 | k-1 | k}+ afin d'aider la macro à produire \PopAt**{e}{L}{k-2 | k-1 | k}.
\end{enumerate}



\subsubsection{Extraction d'une sous-liste}

\begin{enumerate}
	\item \textit{Extraction d'éléments consécutifs.}

	      Lorsque les calculs automatiques ne sont pas faisables, on devra tout indiquer comme dans \verb+\KeepLR{L}{k - 2}{k}+
	      \footnote{
	      	Le nom de la macro vient de \myquote{keep left and right} soit \myquote{garder à droite et à gauche}.
		  }
		  afin d'avoir \KeepLR{L}{k - 2}{k}.
		  
	\item \textit{Extractions juste à droite, ou juste à gauche.}

	      \verb+\KeepL{L}{k}+ permet d'afficher \KeepL{L}{k} et \verb+\KeepR{L}{k}+ permet quant à lui d'écrire \KeepR{L}{k}
	      \footnote{
	      	Les noms des macros viennent de \myquote{keep left} et \myquote{keep right} soit \myquote{garder à gauche} et \myquote{garder à droite}.
		  }.
\end{enumerate}



\subsubsection{Parcourir une liste}

Les macros \verb+\ForInList+ et \verb+\ForInListRev+ facilitent la rédaction de boucle sur une liste parcourue de façon déterministe.

\codeasideoutput{examples/algo-additional/forinlist.tex}

\newpage

\codeasideoutput{examples/algo-additional/forinlist-rev.tex}




%\section{Des outils pour les algorithmes} \label{algo-extra}

\subsection{Boucles sur des entiers consécutifs}

Une boucle \verb+POUR+ peut s'écrire de façon succincte via \verb+\ForRange*+ comme dans l'exemple suivant.

\codeasideoutput{examples/algo-additional/forrange-star.tex}


\vspace{-1em}

Une boucle \verb+POUR+ peut aussi s'écrire de façon non ambigüe via \verb+\ForRange+ non étoilée comme ci-après.


\codeasideoutput{examples/algo-additional/forrange-no-star.tex}


\vspace{-1em}

Pour en finir avec les boucles, l'exemple suivant montre que \verb+\ForRange**{a}{debut}{fin}+ est un alias de \verb+\For{$a \in \CSinterval{debut}{fin}$}+.

\codeasideoutput{examples/algo-additional/forrange-star-star.tex}




\newpage
\section{Ordinogrammes}

\subsection{C'est quoi un ordinogramme} \label{section:flowchart-firstexa}

Les ordinogrammes
\footnote{
    Le mot \myquote{ordinogramme} vient des mots \myquote{ordinateur}, du latin \myquote{ordinare} soit \myquote{mettre en ordre}, et du grec ancien \myquote{gramma} soit \myquote{lettre, écriture}.
}
sont des diagrammes que l'on peut utiliser pour expliquer des algorithmes très simples
\footnote{
    Cet outil pédagogique montre très vite ses limites. Essayez par exemple de tracer un ordinogramme pour expliquer comment résoudre une équation du 2\ieme{} degré.
}.

\medskip


Voici un exemple expliquant comment résoudre $a x^2 + b = 0$, une équation en $x$, lorsque $a \neq 0$ et $b \neq 0$ : le code utilisé est donné plus tard dans la section \ref{section:flowchart-firstexa-code} \emph{(ce code sera très aisé à comprendre une fois lues les sections à venir)}.

\begin{center}
    \small
    \input{examples/flowchart/merly-2nd-degree.tkz}
\end{center}





%\section{Ordinogrammes}

\subsection{L'environnement \texttt{algochart}}

Tous les codes seront placés dans l'environnement \verb+algochart+ qui pour le moment est juste un alias de l'environnement \verb+tikzpicture+ proposé par \verb+TikZ+ qui fait le principal du travail
\footnote{
	\texttt{algochart} vient de la contraction de \myquote{algorithmic} et \myquote{flowchart} soit \myquote{algotithmique} et \myquote{diagramme} en anglais.
}.
\verb+lyalgo+ définit juste quelques styles et quelques macros pour faciliter la saisie des ordinogrammes pour travailler efficacement avec les macros \verb+\node+ et \verb+\path+ proposées par \verb+TikZ+.




%\section{Ordinogrammes}

\subsection{Convention pour les noms des styles et des macros}

Toutes les fonctionnalités proposées par \verb+lyalgo+ seront nommées en minuscule en utilisant toujours le préfixe \verb+ac+ pour \verb+algochart+.




%\section{Ordinogrammes}

\subsection{Les styles proposés}

\begin{frame-gene}[AVERTISSEMENT -- Normes adaptées]
	\centering\itshape
	A la norme officielle, nous avons préféré un style plus percutant

	où les formes des cadres sont bien différenciées.
\end{frame-gene}


% -------------- %


\subsubsection{Entrée et sortie}

L'entrée et la sortie de l'algorithme sont représentés par des ovales comme dans l'exemple ci-après. Par convention, l'entrée se situe tout en haut de l'ordinogramme, et la sortie tout en bas.
Indiquons que \verb+io+ dans \verb+acio+ fait référence à \myquote{input / output} soit \myquote{entrée / sortie} en anglais.

\codeasideoutput{examples/flowchart/showcase/io.tkz}

\vspace{-1em}

Dans le code ci-dessus, nous utilisons le style \verb+acio+ en l'indiquant entre des crochets.
La machinerie \verb+TikZ+ permet de changer localement un réglage comme ci-après où l'on modifie la largeur de l'ellipse pour n'avoir qu'une seule ligne de texte.

\codeasideoutput{examples/flowchart/showcase/io-flatten.tkz}


% -------------- %


\subsubsection{Les instructions}

Dans l'exemple suivant, qui ne nécessite aucun commentaire
\footnote{
	Chercher l'erreur\dots
},
nous avons dû régler à la main la largeur du cadre pour ne pas avoir un retour à la ligne.


\codeasideoutput{examples/flowchart/showcase/instruction.tkz}


% -------------- %


\subsubsection{Les tests conditionnels via un exemple complet}

Nous allons voir comment obtenir le résultat suivant qui contient une structure conditionnelle. Nous allons en profiter pour expliquer comment placer les noeuds les uns par rapport aux autres, et aussi voir comment ajouter des connexions via la macro \verb+\path+ proposée par \verb+TikZ+.


\begin{center}
    \small
    \input{examples/flowchart/if-absolute.tkz}
\end{center}


Voici le code utilisé pour obtenir l'ordinogramme ci-dessus.

\medskip

\justcode{examples/flowchart/if-absolute.tkz}


Donnons des explications sur les points délicats du code précédent.

\begin{enumerate}
	\item \verb+\node[acio] (input) {$n \in \NN$}+
	      
	      \smallskip
	      Ici on définit \verb+input+ comme alias du noeud via \verb+(input)+, un alias utilisable ensuite pour différentes actions graphiques. 

	\medskip
	\item \verb+\node[acif, below of = input] (is-neg) {$n < 0$ ?}+

	      \smallskip
	      Ici on demande de placer le noeud nommé \verb+is-neg+ sous celui nommé \verb+input+ via \verb+below of = input+ où \myquote{below of} se traduit par \myquote{en dessous de} en anglais. Attention au signe égal dans \verb+below of = input+.
	
	\medskip
	\item \verb|\node[acinstr, right] at ($(is-neg) + (2.5,0)$) (neg) {$res \Store (-n)$}|
	      
	      \smallskip
	      Dans cette commande un peu plus mystique, l'emploi de \verb+right+ indique de se placer à droite du dernier noeud.
	      Vient ensuite la cabalistique instruction \verb|at ($(is-neg) + (2.5,0)$)|.
	      Comme \myquote{at} signifie \myquote{à (tel endroit)} en anglais, on comprend que l'on demande de placer le noeud à une certaine position.
	       Il faut alors savoir que pour \verb+TikZ+ l'usage de \verb|($...$)| indique de faire un calcul qui ici est celui d'addition de coordonnées cartésiennes via \verb|(is-neg) + (2.5,0)|. 
	
	\medskip
	\item \verb+\path[aclink] (input) -- (is-neg)+
	      
	      \smallskip
	      Cette instruction plus simple demande de tracer, avec le style \verb+aclink+, un trait entre les noeuds \verb+input+ et \verb+is-neg+.
	
	\medskip
	\item \verb+\path[aclink] (is-neg) -- (neg) \aclabelabove{oui}+
	      
	      \smallskip
	      La nouveauté ici est l'utilisation de la macro \verb+\aclabelabove{oui}+ proposée par \verb+lyalgo+ pour placer du texte au début et au dessus de la connexion car \myquote{above} signifie \myquote{au-dessus} en anglais. 

	\medskip
	\item \verb+\path[aclink] (neg) to[aczigzag] (output);+
	      
	      \smallskip
	      Cette instruction permet d'obtenir \myquote{l'orthopolyligne} 
	      \footnote{
	          Un néologisme ?
	      }
	      de \fbox{$res \Store (-n)$} vers \fbox{$res\vphantom{(-n)}$} .
	      Vous pouvez utiliser \verb+to[aczigzag = {xstart = ... , x = ... , y = ....}]+ pour des réglages personnalisés.
	      Par défaut les clés optionnelles suivent le réglage \verb+xstart = 0mm+ , un décalage au début de la connexion , \verb+x = 3mm+ , un décalage à la fin de la connexion, et \verb+y = 5mm+, un autre décalage à la fin de la connexion.
\end{enumerate}


% -------------- %


\subsubsection{Les boucles}

L'exemple très farfelu qui suit montre comment dessiner une petite boucle en faisant ressortir les instructions liées au fonctionnement de la boucle \emph{(cet effet est impossible à obtenir en mode noir et blanc : voir la section \ref{section:bw-mode} à ce sujet)}.
On voit au passage la limite d'utilisabilité des ordinogrammes car ces derniers ne proposent pas de mise en forme efficace pour les boucles.


\begin{center}
    \small
    \input{examples/flowchart/strange-loop.tkz}
\end{center}


Voici le code que nous avons utilisé. Les seules vraies nouveautés sont l'utilisation du style \verb+acifinstr+ pour mieux visualiser les instructions liées à la boucle, et l'usage de \verb+to[acbackloopleft]+ pour la connexion en retour arrière par la gauche.
Vous pouvez utiliser \verb+to[acbackloopleft = {x = ... }]+  pour régler le décalage vers la gauche.
Par défaut, \verb+x = 5em+.
De façon analogue, on peut utiliser \verb+to[acbackloopright]+ pour un retour arrière par la droite. 

\medskip

\justcode{examples/flowchart/strange-loop.tkz}




%\section{Ordinogrammes}

\subsection{Passer de la couleur au noir et blanc, et vice versa} \label{section:bw-mode}

Pour l'impression papier, n'avoir que du noir et blanc peut rende service. Les commandes \verb+\acusebw+ et \verb+\acusecolor+ permettent d'avoir du noir et blanc ou de la couleur pour tous les ordinogrammes qui suivent l'utilisation de ces macros.

\codeasideoutput{examples/flowchart/showcase/color-fromto-bw.tkz}




%\section{Ordinogrammes}

\subsection{Code du tout premier exemple} \label{section:flowchart-firstexa-code}

Nous (re)donnons la version noir et blanc de l'ordinogramme présenté au début de la section \ref{section:flowchart-firstexa}.

\acusebw
\begin{center}
    \small
    \input{examples/flowchart/merly-2nd-degree.tkz}
\end{center}
\acusecolor

Ce diagramme s'obtient via le code suivant.

\justcode{examples/flowchart/merly-2nd-degree.tkz}




\newpage
\section{Historique}

Nous ne donnons ici qu'un très bref historique de \verb+lyalgo+ côté utilisateur principalement.
Tous les changements sont disponibles uniquement en anglais dans le dossier \verb+change-log+ : voir le code source de \verb+lyalgo+ sur \verb+github+.

\begin{description}[leftmargin=1em]
    \setlength\itemsep{1em}


% --------------- %

%    \item[2019-12-28] Nouvelle version mineure \verb+0.2.0-beta+.
%    \begin{itemize}
%        \item La macro ``\PopAt**`` a cahnég de comportement has changed. %ulti-affectation is used so as to have a coherent behavior.
%
%
%
%
%        **Additional macros:** the macro ``ForRange**`` as an easy way to have a symbolic notation of a ``FOR`` loop.
%
%
%        **¨Doc:** all the ¨latex examples are now organized like the ¨tikz ones.
%
%
%        ==========
%        2019-10-22
%        ==========
%
%        **"Algorithmic flowcharts":** ``to[zigzag]``, ``to[acbackloopleft]`` and ``to[acbackloopright]`` simplify a lot the drawing of "orthopolylines" specific to algorithmic flowcharts.
%    \end{itemize}


% --------------- %

    \item[2019-10-21] Nouvelle version sous mineure \verb+0.1.1-beta+.
    \begin{itemize}
        \item Des nouvelles macros pour les affectations.
        \begin{itemize}
        	\item \verb+\MStore+ et \verb+\MPutIn+ servent à rédiger des affectations multiples en parallèle.

        	\item \verb+\Store*+ et \verb+\Store**+ produisent des écritures symboliques de l'affectation simple via des signes $=$ décorés.
        \end{itemize}

        \item \verb+\CSinterval+ sert à rédiger des intervalles à la sauce informatique.
    \end{itemize}


% --------------- %

    \item[2019-10-19] Nouvelle version mineure \verb+0.1.0-beta+.
    \begin{itemize}
        \item Ajout d'outils pour faciliter le dessin d'ordinogrammes via \verb+TiKz+.
    \end{itemize}


% --------------- %

    \item[2019-10-18] Le documentation a enfin son journal des changements principaux.


% --------------- %

    \item[2019-09-03] Première version \verb+0.0.0-beta+ du package.
\end{description}



\end{document}
