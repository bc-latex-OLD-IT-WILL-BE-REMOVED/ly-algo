%\documentclass[12pt,a4paper]{scrartcl}
\documentclass[12pt,a4paper]{article}

\usepackage[utf8]{inputenc}
\usepackage[T1]{fontenc}


\usepackage{dsfont}

\usepackage[french]{babel,varioref}

\usepackage[top=2cm, bottom=2cm, left=1.5cm, right=1.5cm]{geometry}
\usepackage{enumitem}

\usepackage{color}
\usepackage{hyperref}
\hypersetup{
    colorlinks,
    citecolor=black,
    filecolor=black,
    linkcolor=black,
    urlcolor=black
}

\usepackage{ifplatform}
\usepackage{import}

\usepackage{multicol}


% Our tiny package

\usepackage{lyalgo}


% MISC

\setlength{\parindent}{0cm}


\newenvironment{frame-gene}[1][]{
	\begin{tcolorbox}[title=#1, colbacktitle=black!10!white, colback=white, coltitle=black, breakable,center title, fonttitle=\bfseries\itshape\small]
}{
	\end{tcolorbox}
}



\begin{document}

\title{%
	Le package \texttt{lyalgo}:
	\\
	taper facilement de jolis algorithmes
	\\
	{
		\footnotesize Code source disponible 
		sur \url{https://github.com/bc-latex/ly-algo}.%
	}
	\\
	{
		\footnotesize Version \texttt{0.0.0-beta}
		développée et testée sur \macosxname{}.%
	}
}

\author{Christophe BAL}
\date{2019-10-01}

\maketitle


\vspace{2em}

\hrule

\tableofcontents

\vspace{1.5em}

\hrule

\newpage



% ----------- %


\section{Introduction}

\subimport*{content/}{intro}


% ----------- %



% ----------- %


\section{Écriture \texttt{verbatim}}

\subimport*{content/}{verbatim}


% ----------- %


\section{Algorithmes en language naturel}

\subimport*{content/}{algo}


% ----------- %


%\section{Algorithmes via des diagrammes}
%
%\subimport*{content/}{flowchart}

\end{document}
