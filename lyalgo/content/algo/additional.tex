\subsubsection{Convention en bosses de chameau}

Le package \verb+algorithm2e+ utilise, et abuse
\footnote{
	Ce tyoe de convention est un peu pénible à l'usage. 
},
de la notation en bosses de chameau comme par exemple \verb+\uIf+ et \verb+\Return+ au lieu de \verb+\uif+ et \verb+\return+.
Par cohérence, les nouvelles macros ajoutées par \verb+lyalgo+ utilisent aussi cette convention même si l'auteur aurait préféré proposer \verb+\putin+ et \verb+\forrange+ au lieu de \verb+\PutIn+ et \verb+\ForRange+ par exemple.


% ----------- %


\subsubsection{Affectations}

Les affectations $x \Store 3$ et $3 \PutIn x$ se tapent \verb+$x \Store 3$+ et \verb+$3 \PutIn x$+ respectivement où sont utilisées les macros de type mathématique \verb+\Store+ et \verb+\PutIn+.


% ----------- %


\subsubsection{Boucles}

Une boucle \verb+POUR+ peut s'écrire de façon succincte via \verb+\ForRange*{a}{start}{end}{...}+ pour obtenir ce qui suit.

\begin{algo}[.55]
    \ForRange*{a}{start}{end}{
        \dots\phantom{X}
        \\
        \phantom{\dots}\vspace{-.75em}
    }
\end{algo}


\medskip


Pour une version sans ambiguïté possible, on utilisera \verb+\ForRange{a}{start}{end}{...}+ afin d'obtenir la rédaction plus longue suivante.

\begin{algo}[.55]
    \ForRange{a}{start}{end}{
        \dots\phantom{X}
        \\
        \phantom{\dots}\vspace{-.75em}
    }
\end{algo}


% ----------- %


\subsubsection{Listes}

Via \verb+$\EmptyList$+, on obtiendra une liste vide $\EmptyList$.
