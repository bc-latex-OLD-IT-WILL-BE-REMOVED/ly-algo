La version non étoilée de l'environnement \verb+algo+ ajoute un cadre, comme ci-dessous, afin de rendre plus visibles les algorithmes.

\begin{algo}
    \caption{Un truc bidon}

    \Data{$n \in \mathds{N}^\star$}
    \Result{$\displaystyle \sum_{i = 1}^{n} i$}

    \Actions{
        $s \Store 0$
        \\
        \ForRange{$i$}{$1$}{$n$}{
            $s \Store s + i$
        }
        \Return{$s$}
    }
\end{algo} 


\medskip


Le code utilisé pour obtenir le rendu ci-dessus est le suivant où sont utilisées certaines des macros additionnelles proposées par \verb+lyalgo+ \emph{(voir la section \ref{algo-extra})}.

\begin{frame-gene}[Code \LaTeX]
    \small
    \begin{verbatim}
\begin{algo}
    \caption{Un truc bidon}

    \Data{$n \in \mathds{N}^\star$}
    \Result{$\displaystyle \sum_{i = 1}^{n} i$}

    \Actions{
        $s \Store 0$
        \\
        \ForRange{$i$}{$1$}{$n$}{
            $s \Store s + i$
        }
        \Return{$s$}
    }
\end{algo}
	\end{verbatim}
\end{frame-gene}


\medskip


L'environnement \verb+algo+ propose un argument optionnel pour indiquer la largeur relativement à celle des lignes.
Ainsi  via \verb+\begin{algo}[.45] ... \end{algo}+, on obtient la version suivante bien moins large de l'algorithme précédent.

\begin{algo}[.45]
    \caption{Un truc bidon}

    \Data{$n \in \mathds{N}^\star$}
    \Result{$\displaystyle \sum_{i = 1}^{n} i$}

    \Actions{
        $s \Store 0$
        \\
        \ForRange{$i$}{$1$}{$n$}{
            $s \Store s + i$
        }
        \Return{$s$}
    }
\end{algo} 


\medskip


On peut utiliser un environnement \verb+multicols+ pour un effet sympa.

\begin{multicols}{2}    
\begin{algo}
    \caption{Un truc bidon}

    \Data{$n \in \mathds{N}^\star$}
    \Result{$\displaystyle \sum_{i = 1}^{n} i$}
    \Actions{
        $s \Store 0$
        \\
        \ForRange{$i$}{$1$}{$n$}{
            $s \Store s + i$
        }
        \Return{$s$}
    }
\end{algo} 
\begin{algo}
    \caption{Un truc bidon}

    \Data{$n \in \mathds{N}^\star$}
    \Result{$\displaystyle \sum_{i = 1}^{n} i$}
    \Actions{
        $s \Store 0$
        \\
        \ForRange{$i$}{$1$}{$n$}{
            $s \Store s + i$
        }
        \Return{$s$}
    }
\end{algo}
\end{multicols}


