Le package \verb+algorithm2e+ permet de taper des algorithmes avec une syntaxe simple. La mise en forme par défaut de \verb+algorithm2e+ utilise des flottants, chose qui peut poser des problèmes pour de longs algorithmes ou, c'est plus gênant, pour des algorithmes en bas de page. Dans \verb+lyalgo+, il a été fait le choix de ne pas utiliser de flottants, un choix lié à l'utilisation faite de \verb+lyalgo+ par l'auteur pour rédiger des cours de niveau lycée.


\medskip

Dans la section suivante, nous verrons comment encadrer les algorithmes. Pour l'instant, voyons juste comment taper l'algorithme suivant où tous les mots clés sont en français.


\bigskip


\begin{algo*}
    \caption{Suite de Collatz $(u_k)$ -- Conjecture de Syracuse}

    \Data{$n \in \mathds{N}$}
    \Result{le premier indice $i$ tel que $u_i = 1$ ou $(-1)$ en cas d'échec}

    \BlankLine    % Pour aérer un peu la mise en forme.

    \Actions{
        $i, imax \Store 0, 10^5$
        \\
        $u \Store n$
        \\
        $continuer \Store \top$
        \\
        \While{$continuer = \top$ \And $i \leq imax$}{
            \uIf{$u = 1$}{
                \Comment{C'est gagné !}
                $continuer \Store \bot$
            } \Else {
                \Comment{Calcul du terme suivant}
                \uIf{$u \equiv 0 \,\, [2]$}{
                    $u \Store u / 2$
                    \Comment*{Quotient de la division euclidienne.}
                } \Else {
                    $u \Store 3u + 1$
                }
                $i \Store i + 1$
            }
        }
        \If{$i > imax$}{
            $i \Store (-1)$
        }
        \Return{$i$}
    }
\end{algo*}


\bigskip


La rédaction d'un tel algorithme est facile car il suffit de taper le code suivant proche de ce que pourrait proposer un langage classique de programmation. Le code utilise certaines des macros additionnelles proposées par \verb+lyalgo+ \emph{(voir la section \ref{algo-extra})}. Si besoin voir juste après les lignes de code propres à la syntaxe \verb+algorithm2e+. 


\begin{frame-gene}[Code \LaTeX]
    \small
    \begin{verbatim}
\begin{algo*}
    \caption{Suite de Collatz $(u_k)$ -- Conjecture de Syracuse}

    \Data{$n \in \mathds{N}$}
    \Result{le premier indice $i$ tel que $u_i = 1$ ou $(-1)$ en cas d'échec}

    \BlankLine    % Pour aérer un peu la mise en forme.

    \Actions{
        $i, imax \Store 0, 10^5$
        \\
        $u \Store n$
        \\
        $continuer \Store \top$
        \\
        \While{$continuer = \top$ \And $i \leq imax$}{
            \uIf{$u = 1$}{
                \Comment{C'est gagné !}
                $continuer \Store \bot$
            } \Else {
                \Comment{Calcul du terme suivant}
                \uIf{$u \equiv 0 \,\, [2]$}{
                    $u \Store u / 2$
                    \Comment*{Quotient de la division euclidienne.}
                } \Else {
                    $u \Store 3u + 1$
                }
                $i \Store i + 1$
            }
        }
        \If{$i > imax$}{
            $i \Store (-1)$
        }
        \Return{$i$}
    }
\end{algo*}
    \end{verbatim}
\end{frame-gene}


Le squelette du code précédent est le suivant. 


\begin{frame-gene}[Squelette du code \texttt{algorithm2e}]
    \small
    \begin{verbatim}
\caption{...}

\Data{...}
\Result{...}

\Actions{
    ...
    \While{...}{
        \uIf{...}{
            ...
        } \Else {
            ...
            \uIf{...}{
                ...
            } \Else {
                ...
            }
            ...
        }
    }
    \If{...}{
        ...
    }
    \Return{...}
}
    \end{verbatim}
\end{frame-gene}