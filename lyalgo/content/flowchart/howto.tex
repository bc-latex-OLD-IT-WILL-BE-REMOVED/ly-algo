La construction d'ordinogrammes
\footnote{
	Le mot \myquote{ordinogramme} vient des mots \myquote{ordinateur}, du latin \myquote{ordinare} soit \myquote{mettre en ordre}, et du grec ancien \myquote{gramma} soit \myquote{lettre, écriture}.
	En anglais, on dit \myquote{flow chart}.
}
via TiKz est simplifiée, sans être automatisée, grâce à quelques commandes et styles de mise en forme. Voici un exemple.

\begin{frame-gene}[Résolution dans $\mathds{R}$ de {$a x^2 + b = 0$} si $a \neq 0$]
	\begin{center}
		\subimport*{flowchart/}{trinomial-equation.tikz}
	\end{center}
\end{frame-gene}


???

\begin{frame-gene}[Code \LaTeX{} utilisé\\(uniquement pour l'ordinogramme)]
    \small
    \begin{verbatim}
???
	\end{verbatim}
\end{frame-gene}
