Voici un premier exemple de contenu verbatim où \verb+averb+ vient de \og alternative verbatim \fg. 

\begin{frame-gene}[Code \LaTeX]
	\small
	\begin{verbatim}
\begin{averb*}
  Prix1 = 14 euros
+ Prix2 = 30 euros
-------------------
  Total = 44 euros
\end{averb*} 
	\end{verbatim}
\end{frame-gene}


La mise en forme correspondante est la suivante sans cadre autour.

\begin{frame-gene}[Rendu réel]
	\begin{averb*}
  Prix1 = 14 euros
+ Prix2 = 30 euros
-------------------
  Total = 44 euros
	\end{averb*} 
\end{frame-gene}


\medskip


Il est en fait plus pratique de pouvoir taper quelque chose comme ci-dessous avec un cadre autour.

\begin{averb}[Une sortie console]
  Prix1 = 14 euros
+ Prix2 = 30 euros
-------------------
  Total = 44 euros
\end{averb} 


Le contenu précédent s'obtient via le code suivant.

\begin{frame-gene}[Code \LaTeX]
	\small
	\begin{verbatim}
\begin{averb}[Une sortie console]
  Prix1 = 14 euros
+ Prix2 = 30 euros
-------------------
  Total = 44 euros
\end{averb} 
	\end{verbatim} 
\end{frame-gene}


\medskip

Notez bien que c'est la version étoilée de \verb+averb+ qui en fait le moins. Ce principe sera aussi suivi pour les algorithmes.

\vfill
\null
