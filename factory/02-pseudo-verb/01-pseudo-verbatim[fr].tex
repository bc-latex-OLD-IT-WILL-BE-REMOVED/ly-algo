\documentclass[12pt,a4paper]{article}

\makeatletter
	\usepackage[utf8]{inputenc}
\usepackage[T1]{fontenc}

\usepackage{dsfont}

\usepackage[french]{babel,varioref}

\usepackage[top=2cm, bottom=2cm, left=1.5cm, right=1.5cm]{geometry}
\usepackage{enumitem}

\usepackage{color}
\usepackage{hyperref}
\hypersetup{
    colorlinks,
    citecolor=black,
    filecolor=black,
    linkcolor=black,
    urlcolor=black
}

\usepackage{ifplatform}
\usepackage{import}

\usepackage{multicol}

\usepackage{tcolorbox}

\usepackage{amsthm}

\usepackage{ifplatform}

\usepackage{cbdevtool}
\usepackage{lymath}

% MISC

\setlength{\parindent}{0cm}
\setlist{noitemsep}


\theoremstyle{definition}
\newtheorem*{remark}{Remarque}


\usepackage[raggedright]{titlesec}

\titleformat{\paragraph}[hang]{\normalfont\normalsize\bfseries}{\theparagraph}{1em}{}
\titlespacing*{\paragraph}{0pt}{3.25ex plus 1ex minus .2ex}{0.5em}

\setlength{\parindent}{0cm}

\newenvironment{frame-gene}[1][]{
	\begin{tcolorbox}[
		title        = #1, 
		colbacktitle = black!10!white, 
		colback      = white, 
		coltitle     = black,
		fonttitle    = \bfseries\itshape\small, 
		breakable,
		center title]
}{
	\end{tcolorbox}
}

\newcommand\myquote[1]{{\itshape \og #1 \fg}}

\newcommand\squaremacro{$x^2$}

	% == PACKAGES USED == %

\usepackage{tcolorbox}
\usepackage{xparse}


% == DEFINITIONS == %

\tcbuselibrary{breakable}


% Source used : 
%	* https://tex.stackexchange.com/a/26873/6880

\ExplSyntaxOn

\NewDocumentCommand{\instringTF}{mmmm}
 {
  \oleks_instring:nnnn { #1 } { #2 } { #3 } { #4 }
 }

\tl_new:N \l__oleks_instring_test_tl

\cs_new_protected:Nn \oleks_instring:nnnn
 {
  \tl_set:Nn \l__oleks_instring_test_tl { #1 }
  \regex_match:nnTF { \u{l__oleks_instring_test_tl} } { #2 } { #3 } { #4 }
 }

\ExplSyntaxOff

	
	\usepackage{01-pseudo-verbatim}
\makeatother



\begin{document}

\section{Écriture \texttt{pseudo-verbatim}}

En complément à l'environnement \verb+verbatim+ est proposé l'environnement \verb+pseudoverb+, pour \og pseudo verbatim \fg, qui permet d'écrire du contenu presque verbatim : ci-après, la macro \verb+\squaremacro+ définie par \verb+\newcommand\squaremacro{$x^2$}+ est interprétée mais pas la formule de mathématiques. 

\begin{frame-gene}[Code \LaTeX]
	\small
	\begin{verbatim}
\begin{pseudoverb*}
  Prix1 = 14 euros 
+ Prix2 = 30 euros
-------------------
  Total = 44 euros

========
Remarque
========
Attention car les macros comme \squaremacro{} sont interprétées mais pas les formules
de maths comme $x^2$ !
\end{pseudoverb*} 
	\end{verbatim}
\end{frame-gene}


La mise en forme correspondante est la suivante sans cadre autour.

\begin{frame-gene}[Rendu réel\\ATTENTION ! Le cadre ne fait pas partie de la mise en forme.]
	\begin{pseudoverb*}
  Prix1 = 14 euros
+ Prix2 = 30 euros
-------------------
  Total = 44 euros

========
Remarque
========
Attention car les macros comme \squaremacro{} sont interprétées mais pas les formules
de maths comme $x^2$ !
	\end{pseudoverb*} 
\end{frame-gene}


\medskip


Il est en fait plus pratique de pouvoir taper quelque chose comme ci-dessous avec un cadre autour où le titre est un argument obligatoire \emph{(voir plus bas comment ne pas avoir de titre)}.

\begin{pseudoverb}{Une sortie console}
  Prix1 = 14 euros
+ Prix2 = 30 euros
-------------------
  Total = 44 euros
\end{pseudoverb} 


Le contenu précédent s'obtient via le code suivant.

\begin{frame-gene}[Code \LaTeX]
	\small
	\begin{verbatim}
\begin{pseudoverb}{Une sortie console}
  Prix1 = 14 euros
+ Prix2 = 30 euros
-------------------
  Total = 44 euros
\end{pseudoverb} 
	\end{verbatim} 
\end{frame-gene}


Finissons avec une version bien moins large et sans titre de la sortie console ci-dessus. Le principe est de donner un titre vide via \verb+{}+, c'est obligatoire, et en utilisant l'unique argument optionnel pour indiquer la largeur relativement à celle des lignes. En utilisant \verb+\begin{pseudoverb}[.275]{}+ au lieu \verb+\begin{pseudoverb}{Une sortie console}+, le code précédent nous donne ce qui suit.


\begin{pseudoverb}[.275]{}
  Prix1 = 14 euros
+ Prix2 = 30 euros
-------------------
  Total = 44 euros
\end{pseudoverb} 


\paragraph{A retenir.} C'est la version étoilée de \verb+pseudoverb+ qui en fait le moins. Ce principe sera aussi suivi pour les algorithmes.

\end{document}
