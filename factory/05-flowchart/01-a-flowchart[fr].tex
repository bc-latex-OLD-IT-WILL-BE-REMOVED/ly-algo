\documentclass[12pt,a4paper]{article}

\makeatletter
    \usepackage[utf8]{inputenc}
\usepackage[T1]{fontenc}

\usepackage{dsfont}

\usepackage[french]{babel,varioref}

\usepackage[top=2cm, bottom=2cm, left=1.5cm, right=1.5cm]{geometry}
\usepackage{enumitem}

\usepackage{color}
\usepackage{hyperref}
\hypersetup{
    colorlinks,
    citecolor=black,
    filecolor=black,
    linkcolor=black,
    urlcolor=black
}

\usepackage{ifplatform}
\usepackage{import}

\usepackage{multicol}

\usepackage{tcolorbox}

\usepackage{amsthm}

\usepackage{ifplatform}

\usepackage{cbdevtool}
\usepackage{lymath}

% MISC

\setlength{\parindent}{0cm}
\setlist{noitemsep}


\theoremstyle{definition}
\newtheorem*{remark}{Remarque}


\usepackage[raggedright]{titlesec}

\titleformat{\paragraph}[hang]{\normalfont\normalsize\bfseries}{\theparagraph}{1em}{}
\titlespacing*{\paragraph}{0pt}{3.25ex plus 1ex minus .2ex}{0.5em}

\setlength{\parindent}{0cm}

\newenvironment{frame-gene}[1][]{
	\begin{tcolorbox}[
		title        = #1, 
		colbacktitle = black!10!white, 
		colback      = white, 
		coltitle     = black,
		fonttitle    = \bfseries\itshape\small, 
		breakable,
		center title]
}{
	\end{tcolorbox}
}

\newcommand\myquote[1]{{\itshape \og #1 \fg}}

\newcommand\squaremacro{$x^2$}

    % == PACKAGES USED == %

\usepackage{tcolorbox}
\usepackage{xparse}


% == DEFINITIONS == %

\tcbuselibrary{breakable}


% Source used : 
%	* https://tex.stackexchange.com/a/26873/6880

\ExplSyntaxOn

\NewDocumentCommand{\instringTF}{mmmm}
 {
  \oleks_instring:nnnn { #1 } { #2 } { #3 } { #4 }
 }

\tl_new:N \l__oleks_instring_test_tl

\cs_new_protected:Nn \oleks_instring:nnnn
 {
  \tl_set:Nn \l__oleks_instring_test_tl { #1 }
  \regex_match:nnTF { \u{l__oleks_instring_test_tl} } { #2 } { #3 } { #4 }
 }

\ExplSyntaxOff


    \usepackage{01-a-flowchart}
\makeatother


\newcommand\Store{\leftarrow}
\newcommand\List[1]{[\,#1\,]}
\newcommand\EmptyList{\List{}}


\begin{document}

\newpage
\section{Ordinogrammes}

\subsection{C'est quoi un ordinogramme} \label{section:flowchart-firstexa}

Les ordinogrammes
\footnote{
    Le mot \myquote{ordinogramme} vient des mots \myquote{ordinateur}, du latin \myquote{ordinare} soit \myquote{mettre en ordre}, et du grec ancien \myquote{gramma} soit \myquote{lettre, écriture}.
}
sont des diagrammes que l'on peut utiliser pour expliquer des algorithmes très simples
\footnote{
    Cet outil pédagogique montre très vite ses limites. Essayez par exemple de tracer un ordinogramme pour expliquer comment résoudre une équation du 2\ieme{} degré.
}.

\medskip


Voici un exemple expliquant comment résoudre $a x^2 + b = 0$, une équation en $x$, lorsque $a \neq 0$ et $b \neq 0$ : le code utilisé est donné plus tard dans la section \ref{section:flowchart-firstexa-code} \emph{(ce code sera très aisé à comprendre une fois lues les sections à venir)}.

\begin{center}
    \small
    \input{examples/flowchart/merly-2nd-degree.tkz}
\end{center}


\end{document}
