\documentclass[12pt,a4paper]{article}

\makeatletter
	\usepackage[utf8]{inputenc}
\usepackage[T1]{fontenc}

\usepackage{dsfont}

\usepackage[french]{babel,varioref}

\usepackage[top=2cm, bottom=2cm, left=1.5cm, right=1.5cm]{geometry}
\usepackage{enumitem}

\usepackage{color}
\usepackage{hyperref}
\hypersetup{
    colorlinks,
    citecolor=black,
    filecolor=black,
    linkcolor=black,
    urlcolor=black
}

\usepackage{ifplatform}
\usepackage{import}

\usepackage{multicol}

\usepackage{tcolorbox}

\usepackage{amsthm}

\usepackage{ifplatform}

\usepackage{cbdevtool}
\usepackage{lymath}

% MISC

\setlength{\parindent}{0cm}
\setlist{noitemsep}


\theoremstyle{definition}
\newtheorem*{remark}{Remarque}


\usepackage[raggedright]{titlesec}

\titleformat{\paragraph}[hang]{\normalfont\normalsize\bfseries}{\theparagraph}{1em}{}
\titlespacing*{\paragraph}{0pt}{3.25ex plus 1ex minus .2ex}{0.5em}

\setlength{\parindent}{0cm}

\newenvironment{frame-gene}[1][]{
	\begin{tcolorbox}[
		title        = #1, 
		colbacktitle = black!10!white, 
		colback      = white, 
		coltitle     = black,
		fonttitle    = \bfseries\itshape\small, 
		breakable,
		center title]
}{
	\end{tcolorbox}
}

\newcommand\myquote[1]{{\itshape \og #1 \fg}}

\newcommand\squaremacro{$x^2$}

	% == PACKAGES USED == %

\usepackage{tcolorbox}
\usepackage{xparse}


% == DEFINITIONS == %

\tcbuselibrary{breakable}


% Source used : 
%	* https://tex.stackexchange.com/a/26873/6880

\ExplSyntaxOn

\NewDocumentCommand{\instringTF}{mmmm}
 {
  \oleks_instring:nnnn { #1 } { #2 } { #3 } { #4 }
 }

\tl_new:N \l__oleks_instring_test_tl

\cs_new_protected:Nn \oleks_instring:nnnn
 {
  \tl_set:Nn \l__oleks_instring_test_tl { #1 }
  \regex_match:nnTF { \u{l__oleks_instring_test_tl} } { #2 } { #3 } { #4 }
 }

\ExplSyntaxOff

	% == PACKAGES USED == %

\usepackage[french, vlined]{algorithm2e}


% == DEFINITIONS == %

% Algo - Frames

\newenvironment{algo*}
	{\begin{algorithm}[H]}
	{\end{algorithm}}


\newenvironment{algo}[1][1]{
	\centering
	\begin{tcolorbox}[
		colback = white,
		width=#1\linewidth,
		breakable
	]
	\begin{algo*}
}{
	\end{algo*}
	\vspace{-0.5em}
	\end{tcolorbox}
}


\newcommand\addalgoblank[1][]{
   \ifthenelse{ \equal{#1}{} }
      	{\vspace{.2em}}
      	{\foreach \n in {0,...,#1}{\vspace{.2em}}}
}

	\usepackage{04-keywords}
\makeatother



\begin{document}

%\section{Algorithmes en langage naturel}

\subsection{Un premier ensemble de macros additionnelles ou francisées}

\begin{frame-gene}[À SAVOIR -- Le préfixe \texttt{u}]
	\centering\itshape
	Certains macros peuvent être préfixées par un \verb+u+ pour \myquote{unclosed} qui signifie \myquote{non fermé}.
	
	Ceci sert à ne pas refermer un bloc via un trait horizontal.
\end{frame-gene}


% == All extra words - START == %
\subsubsection{Entrée / Sortie}

Nous donnons ci-dessous les versions au singulier de tous les mots disponibles de type \myquote{entrée / sortie}.
Excepté pour \verb+\InState+ et \verb+\OutState+, toutes les autres macros ont une version pour le pluriel obtenu en rajoutant un \verb+s+ à la fin du nom de la macro.
Par exemple, le pluriel de \verb+\In+ s'obtient via \verb+\Ins+.


\begin{multicols}{2}
    \centering
    \begin{frame-gene}[Code \LaTeX]
\begin{verbatim}
\begin{algo}
  \In{donnée 1}
  \Out{donnée 2}
\end{algo}
\end{verbatim}
    \end{frame-gene}
    \vfill\null
    \columnbreak
    \textit{Mise en forme correspondante.}
\begin{algo}
  \In{donnée 1}
  \Out{donnée 2}
\end{algo}
    \vfill\null
\end{multicols}


\begin{multicols}{2}
    \centering
    \begin{frame-gene}[Code \LaTeX]
\begin{verbatim}
\begin{algo}
  \Data{donnée 1}
  \Result{donnée 2}
\end{algo}
\end{verbatim}
    \end{frame-gene}
    \vfill\null
    \columnbreak
    \textit{Mise en forme correspondante.}
\begin{algo}
  \Data{donnée 1}
  \Result{donnée 2}
\end{algo}
    \vfill\null
\end{multicols}


\begin{multicols}{2}
    \centering
    \begin{frame-gene}[Code \LaTeX]
\begin{verbatim}
\begin{algo}
  \InState{donnée 1}
  \OutState{donnée 2}
\end{algo}
\end{verbatim}
    \end{frame-gene}
    \vfill\null
    \columnbreak
    \textit{Mise en forme correspondante.}
\begin{algo}
  \InState{donnée 1}
  \OutState{donnée 2}
\end{algo}
    \vfill\null
\end{multicols}


\begin{multicols}{2}
    \centering
    \begin{frame-gene}[Code \LaTeX]
\begin{verbatim}
\begin{algo}
  \PreCond{donnée 1}
  \PostCond{donnée 2}
\end{algo}
\end{verbatim}
    \end{frame-gene}
    \vfill\null
    \columnbreak
    \textit{Mise en forme correspondante.}
\begin{algo}
  \PreCond{donnée 1}
  \PostCond{donnée 2}
\end{algo}
    \vfill\null
\end{multicols}


\subsubsection{Bloc principal}

Voici comment indiquer le bloc principal d'instructions avec deux textes au choix pour le moment.


\begin{multicols}{2}
    \centering
    \begin{frame-gene}[Code \LaTeX]
\begin{verbatim}
\begin{algo}
  \Actions{Instruction 1}
  \Begin{Instruction 2}
\end{algo}
\end{verbatim}
    \end{frame-gene}
    \vfill\null
    \columnbreak
    \textit{Mise en forme correspondante.}
\begin{algo}
  \Actions{Instruction 1}
  \Begin{Instruction 2}
\end{algo}
    \vfill\null
\end{multicols}


\subsubsection{Boucles \TTfor{} et \TTwhile{}}

Voici les boucles de type \TTfor{} et \TTwhile{} proposées par le package.

\newpage


\begin{multicols}{2}
    \centering
    \begin{frame-gene}[Code \LaTeX]
\begin{verbatim}
\begin{algo}
  \For{$i \in uneliste$}{
    Instruction 1
  }
  \ForAll{$i \in uneliste$}{
    Instruction 2
  }
  \ForEach{$i \in uneliste$}{
    Instruction 3
  }
  \While{$i \in uneliste$}{
    Instruction 4
  }
\end{algo}
\end{verbatim}
    \end{frame-gene}
    \vfill\null
    \columnbreak
    \textit{Mise en forme correspondante.}
\begin{algo}
  \For{$i \in uneliste$}{
    Instruction 1
  }
  \ForAll{$i \in uneliste$}{
    Instruction 2
  }
  \ForEach{$i \in uneliste$}{
    Instruction 3
  }
  \While{$i \in uneliste$}{
    Instruction 4
  }
\end{algo}
    \vfill\null
\end{multicols}


\subsubsection{Boucles \TTrepeat{}}

Voici comment rédiger une boucle du type \TTrepeat{}.


\begin{multicols}{2}
    \centering
    \begin{frame-gene}[Code \LaTeX]
\begin{verbatim}
\begin{algo}
  \Repeat{$i \in uneliste$}{
    Instruction 
  }
\end{algo}
\end{verbatim}
    \end{frame-gene}
    \vfill\null
    \columnbreak
    \textit{Mise en forme correspondante.}
\begin{algo}
  \Repeat{$i \in uneliste$}{
    Instruction 
  }
\end{algo}
    \vfill\null
\end{multicols}


\subsubsection{Disjonction de cas \TTswitch{}}

La syntaxe pour les blocs conditionnels du type \TTswitch{} ne pose pas de difficultés de rédaction.


\begin{multicols}{2}
    \centering
    \begin{frame-gene}[Code \LaTeX]
\begin{verbatim}
\begin{algo}
  \Switch{$i$}{
    \uCase{$i = 0$}{Instruction 1}
    \uCase{$i = 1$}{Instruction 2}
    \Case{$i = 2$}{Instruction 3}
  }
\end{algo}
\end{verbatim}
    \end{frame-gene}
    \vfill\null
    \columnbreak
    \textit{Mise en forme correspondante.}
\begin{algo}
  \Switch{$i$}{
    \uCase{$i = 0$}{Instruction 1}
    \uCase{$i = 1$}{Instruction 2}
    \Case{$i = 2$}{Instruction 3}
  }
\end{algo}
    \vfill\null
\end{multicols}


\subsubsection{Disjonction conditionnelle \TTif{}}

Les blocs conditionnels \TTif{} se rédigent très naturellement.


\begin{multicols}{2}
    \centering
    \begin{frame-gene}[Code \LaTeX]
\begin{verbatim}
\begin{algo}
  \uIf{$i = 0$}{
    Instruction 1
  }
  \uElseIf{$i = 1$}{
    Instruction 2
  }
  \Else{
    Instruction 3
  }
\end{algo}
\end{verbatim}
    \end{frame-gene}
    \vfill\null
    \columnbreak
    \textit{Mise en forme correspondante.}
\begin{algo}
  \uIf{$i = 0$}{
    Instruction 1
  }
  \uElseIf{$i = 1$}{
    Instruction 2
  }
  \Else{
    Instruction 3
  }
\end{algo}
    \vfill\null
\end{multicols}


\subsubsection{Diverses commandes}

Pour finir voici un ensemble de mots supplémentaires qui pourront vous rendre service. Le préfixe \verb+m+ permet d'utiliser des versions maculines des textes proposés.


\begin{multicols}{2}
    \centering
    \begin{frame-gene}[Code \LaTeX]
\begin{verbatim}
\begin{algo}
  A \And B \Or C
  \\ \Return RÉSULTAT
  \\ \Ask "Quelque chose"
  \\ \Print "Quelque chose"
  \\ $k$ \From $1$ \To $n$
  \\ $k$ \ComingFrom $1$ \GoingTo $n$
  \\ $e$ \InThis $\{ 1 , 4 , 16 \}$
  \\ $L$ \LToR
  \\ $L$ \LToRm
  \\ $L$ \RToL
  \\ $L$ \RToLm
\end{algo}
\end{verbatim}
    \end{frame-gene}
    \vfill\null
    \columnbreak
    \textit{Mise en forme correspondante.}
\begin{algo}
  A \And B \Or C
  \\ \Return RÉSULTAT
  \\ \Ask "Quelque chose"
  \\ \Print "Quelque chose"
  \\ $k$ \From $1$ \To $n$
  \\ $k$ \ComingFrom $1$ \GoingTo $n$
  \\ $e$ \InThis $\{ 1 , 4 , 16 \}$
  \\ $L$ \LToR
  \\ $L$ \LToRm
  \\ $L$ \RToL
  \\ $L$ \RToLm
\end{algo}
    \vfill\null
\end{multicols}


% == All extra words - END == %


% ---------------- %


\subsection{Citer les outils de base en algorithmique}

Pour faciliter la rédaction de textes sur les algorithmes, des macros standardisent l'impression des noms des outils classiques de contrôle.
Dans les exemples qui suivent, les préfixes \verb+TT+ et \verb+AL+ font référence à \myquote{True Type} pour une police à chasse fixe, et à \myquote{AL-gorithme} pour une écriture similaire à celle utilisée dans les algorithmes.


% == Main tools - START == %

\begin{center}
	Liste des commandes de type \myquote{True Type}.
\end{center}

\begin{enumerate}
    \item \verb+\TTif+ donne \TTif.
    \item \verb+\TTfor+ donne \TTfor.
    \item \verb+\TTwhile+ donne \TTwhile.
    \item \verb+\TTrepeat+ donne \TTrepeat.
    \item \verb+\TTswitch+ donne \TTswitch.
\end{enumerate}

\begin{center}
	Liste des commandes de type \myquote{algorithme}.
\end{center}

\begin{enumerate}
    \item \verb+\ALif+ donne \ALif.
    \item \verb+\ALfor+ donne \ALfor.
    \item \verb+\ALwhile+ donne \ALwhile.
    \item \verb+\ALrepeat+ donne \ALrepeat.
    \item \verb+\ALswitch+ donne \ALswitch.
\end{enumerate}
% == Main tools - END == %

\end{document}
