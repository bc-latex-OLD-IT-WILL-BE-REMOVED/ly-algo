\documentclass[12pt,a4paper]{article}

\makeatletter
	\usepackage[utf8]{inputenc}
\usepackage[T1]{fontenc}

\usepackage{dsfont}

\usepackage[french]{babel,varioref}

\usepackage[top=2cm, bottom=2cm, left=1.5cm, right=1.5cm]{geometry}
\usepackage{enumitem}

\usepackage{color}
\usepackage{hyperref}
\hypersetup{
    colorlinks,
    citecolor=black,
    filecolor=black,
    linkcolor=black,
    urlcolor=black
}

\usepackage{ifplatform}
\usepackage{import}

\usepackage{multicol}

\usepackage{tcolorbox}

\usepackage{amsthm}

\usepackage{ifplatform}

\usepackage{cbdevtool}
\usepackage{lymath}

% MISC

\setlength{\parindent}{0cm}
\setlist{noitemsep}


\theoremstyle{definition}
\newtheorem*{remark}{Remarque}


\usepackage[raggedright]{titlesec}

\titleformat{\paragraph}[hang]{\normalfont\normalsize\bfseries}{\theparagraph}{1em}{}
\titlespacing*{\paragraph}{0pt}{3.25ex plus 1ex minus .2ex}{0.5em}

\setlength{\parindent}{0cm}

\newenvironment{frame-gene}[1][]{
	\begin{tcolorbox}[
		title        = #1, 
		colbacktitle = black!10!white, 
		colback      = white, 
		coltitle     = black,
		fonttitle    = \bfseries\itshape\small, 
		breakable,
		center title]
}{
	\end{tcolorbox}
}

\newcommand\myquote[1]{{\itshape \og #1 \fg}}

\newcommand\squaremacro{$x^2$}

	% == PACKAGES USED == %

\usepackage{tcolorbox}
\usepackage{xparse}


% == DEFINITIONS == %

\tcbuselibrary{breakable}


% Source used : 
%	* https://tex.stackexchange.com/a/26873/6880

\ExplSyntaxOn

\NewDocumentCommand{\instringTF}{mmmm}
 {
  \oleks_instring:nnnn { #1 } { #2 } { #3 } { #4 }
 }

\tl_new:N \l__oleks_instring_test_tl

\cs_new_protected:Nn \oleks_instring:nnnn
 {
  \tl_set:Nn \l__oleks_instring_test_tl { #1 }
  \regex_match:nnTF { \u{l__oleks_instring_test_tl} } { #2 } { #3 } { #4 }
 }

\ExplSyntaxOff


	\usepackage{02-frame}
	
	\newcommand\ForRange{\@ifstar{\@ForRange@star}{\@ForRange@no@star}}

	\newcommand\@ForRange@no@star[4]{
		\For{#1 allant de \text{$#2$} à \text{$#3$}}{#4}
	}

	\newcommand\@ForRange@star[4]{
		\@ForRange@no@star{#1}{#2}{#3}{#4}
	}
\makeatother


\newcommand\Store\leftarrow

\SetKwInput{Data}{Donnée}
\SetKwInput{Result}{Résultat}
\SetKwBlock{Actions}{Actions}{}


\begin{document}

%\section{Algorithmes en language naturel}

\subsection{Des algorithmes encadrés}

La version non étoilée de l'environnement \verb+algo+ ajoute un cadre, comme ci-dessous, afin de rendre plus visibles les algorithmes.

\begin{algo}
    \caption{Un truc bidon}

    \Data{$n \in \NNs$}
    \addalgoblank
    \Result{$\dsum_{i = 1}^{n} i$}

    \Actions{
        $s \Store 0$
        \\
        \ForRange{$i$}{$1$}{$n$}{
            $s \Store s + i$
        }
        \Return{$s$}
    }
\end{algo} 


\medskip


Le code utilisé pour obtenir le rendu précédent est le suivant où sont utilisées certaines des macros additionnelles proposées par \verb+lyalgo+ \emph{(voir la section \ref{algo-extra})} où les macros \verb+\NNs+ et \verb+\dsum+ viennent du package \verb+lymath+.

\begin{frame-gene}[Code \LaTeX]
    \small
    \begin{verbatim}
\begin{algo}
    \caption{Un truc bidon}
    \Data{$n \in \NNs$}
    \addalgoblank
    \Result{$\dsum_{i = 1}^{n} i$}
    \Actions{
        $s \Store 0$
        \\
        \ForRange{$i$}{$1$}{$n$}{
            $s \Store s + i$
        }
        \Return{$s$}
    }
\end{algo}
	\end{verbatim}
\end{frame-gene}


\medskip


L'environnement \verb+algo+ propose un argument optionnel pour indiquer la largeur relativement à celle des lignes.
Ainsi  via \verb+\begin{algo}[.45] ... \end{algo}+, on obtient la version suivante bien moins large de l'algorithme précédent.

\begin{algo}[.45]
    \caption{Un truc bidon}

    \Data{$n \in \NNs$}
    \addalgoblank
    \Result{$\dsum_{i = 1}^{n} i$}

    \Actions{
        $s \Store 0$
        \\
        \ForRange*{$i$}{$1$}{$n$}{
            $s \Store s + i$
        }
        \Return{$s$}
    }
\end{algo} 


\medskip


On peut utiliser un environnement \verb+multicols+ pour un effet sympa.

\begin{multicols}{2}    
\begin{algo}
    \caption{Un truc bidon}

    \Data{$n \in \NNs$}
    \addalgoblank
    \Result{$\dsum_{i = 1}^{n} i$}
    \Actions{
        $s \Store 0$
        \\
        \ForRange*{$i$}{$1$}{$n$}{
            $s \Store s + i$
        }
        \Return{$s$}
    }
\end{algo}


\begin{algo}
    \caption{Un truc bidon}

    \Data{$n \in \NNs$}
    \addalgoblank
    \Result{$\dsum_{i = 1}^{n} i$}
    \Actions{
        $s \Store 0$
        \\
        \ForRange{$i$}{$1$}{$n$}{
            $s \Store s + i$
        }
        \Return{$s$}
    }
\end{algo}
\end{multicols}

\end{document}