\documentclass[12pt,a4paper]{article}

\makeatletter
	\usepackage[utf8]{inputenc}
\usepackage[T1]{fontenc}

\usepackage{dsfont}

\usepackage[french]{babel,varioref}

\usepackage[top=2cm, bottom=2cm, left=1.5cm, right=1.5cm]{geometry}
\usepackage{enumitem}

\usepackage{color}
\usepackage{hyperref}
\hypersetup{
    colorlinks,
    citecolor=black,
    filecolor=black,
    linkcolor=black,
    urlcolor=black
}

\usepackage{ifplatform}
\usepackage{import}

\usepackage{multicol}

\usepackage{tcolorbox}

\usepackage{amsthm}

\usepackage{ifplatform}

\usepackage{cbdevtool}
\usepackage{lymath}

% MISC

\setlength{\parindent}{0cm}
\setlist{noitemsep}


\theoremstyle{definition}
\newtheorem*{remark}{Remarque}


\usepackage[raggedright]{titlesec}

\titleformat{\paragraph}[hang]{\normalfont\normalsize\bfseries}{\theparagraph}{1em}{}
\titlespacing*{\paragraph}{0pt}{3.25ex plus 1ex minus .2ex}{0.5em}

\setlength{\parindent}{0cm}

\newenvironment{frame-gene}[1][]{
	\begin{tcolorbox}[
		title        = #1, 
		colbacktitle = black!10!white, 
		colback      = white, 
		coltitle     = black,
		fonttitle    = \bfseries\itshape\small, 
		breakable,
		center title]
}{
	\end{tcolorbox}
}

\newcommand\myquote[1]{{\itshape \og #1 \fg}}

\newcommand\squaremacro{$x^2$}

	% == PACKAGES USED == %

\usepackage{tcolorbox}
\usepackage{xparse}


% == DEFINITIONS == %

\tcbuselibrary{breakable}


% Source used : 
%	* https://tex.stackexchange.com/a/26873/6880

\ExplSyntaxOn

\NewDocumentCommand{\instringTF}{mmmm}
 {
  \oleks_instring:nnnn { #1 } { #2 } { #3 } { #4 }
 }

\tl_new:N \l__oleks_instring_test_tl

\cs_new_protected:Nn \oleks_instring:nnnn
 {
  \tl_set:Nn \l__oleks_instring_test_tl { #1 }
  \regex_match:nnTF { \u{l__oleks_instring_test_tl} } { #2 } { #3 } { #4 }
 }

\ExplSyntaxOff

	\usepackage{../03-algo-basic/02-frame}
	\usepackage{02-affectation}
	
	\usepackage{04-list}
\makeatother


\newcommand\InThis{dans}
\newcommand\LToR{parcourue de gauche à droite}
\newcommand\RToL{parcourue de droite à gauche}


\begin{document}

%\section{Des outils pour les algorithmes} \label{algo-extra}

\subsection{Listes}

\begin{frame-gene}[AVERTISSEMENT -- Premier indice]
	\centering\itshape
	Pour le package, les indices des listes commencent toujours à un. 
\end{frame-gene}


\subsubsection{Opérations de base.}

Voici les premières macros pour travailler avec des listes c'est à dire des tableaux de taille modifiable.

\begin{enumerate}
	\item \textit{Liste vide.}

		  \verb+\EmptyList+ imprime une liste vide $\EmptyList$.


	\item \textit{Liste en extension.}

		  \verb+\List{4 ; 7 ; 7 ; -1}+ produit $\List{4 ; 7 ; 7 ; -1}$.


	\item \textit{Le $k$\ieme{} élément d'une liste.}

		  \verb+\ListElt{L}{1}+ produit $\ListElt{L}{1}$.


	\item \textit{La sous-liste des éléments jusqu'à celui à la position $k$.}

		  \verb+\ListUntil{L}{2}+ produit $\ListUntil{L}{2}$.


	\item \textit{La sous-liste des éléments à partir de celui à la position $k$.}

		  \verb+\ListFrom{L}{2}+ produit $\ListFrom{L}{2}$.


	\item \textit{Concaténer deux listes.}

		  \verb+\AddList+ est l'opérateur binaire $\AddList$ qui permet d'indiquer la concaténation de deux listes.


	\item \textit{Taille ou longueur d'une liste.}

		  La macro \verb+\Len(L)+ produit $\Len(L)$.
\end{enumerate}



\subsubsection{Modifier une liste -- Versions textuelles}

\begin{enumerate}
	\item \textit{Ajout d'un nouvel élément à droite.}

	      \verb+\Append{L}{5}+ produit \myquote{\Append{L}{5}}
	      \footnote{
		       Le verbe anglais \myquote{append} signifie \myquote{ajouter}.
		  }.


	\item \textit{Ajout d'un nouvel élément à gauche.}

	      \verb+\Prepend{L}{5}+ produit \myquote{\Prepend{L}{5}}
	      \footnote{
		       Le verbe anglais \myquote{prepend} signifie \myquote{préfixer}.
		  }.


	\item \textit{Extraction d'un élément.}

	      \verb+\PopAt{L}{3}+ produit \myquote{\PopAt{L}{3}}.
\end{enumerate}



\subsubsection{Modifier une liste -- Versions POO}

Les versions étoilées des macros précédentes fournissent une autre mise en forme à la fois concise et aisée à comprendre
\footnote{
	L'opérateur point \POOpoint{} est défini dans la macro \texttt{\textbackslash{}POOpoint}. 
	Ceci permet de personnaliser facilement cet opérateur.
}
avec une syntaxe de type POO
\footnote{
	\myquote{POO} est l'acronyme de \myquote{Programmation Orientée Objet}.
}.


\begin{enumerate}
	\item \textit{Ajout d'un nouvel élément à droite.}

	      \verb+\Append*{L}{5}+ fournit \Append*{L}{5}.


	\item \textit{Ajout d'un nouvel élément à gauche.}

	      \verb+\Prepend*{L}{5}+ fournit \Prepend*{L}{5}.


	\item \textit{Extraction d'un élément.}

	      \verb+\PopAt*{L}{3}+ fournit \PopAt*{L}{3}.
\end{enumerate}



\subsubsection{Modifier une liste -- Versions symboliques}

Des versions doublement étoilées permettent d'obtenir des notations symboliques qui sont très efficaces lorsque l'on rédige les algorithmes à la main
\footnote{
	L'opérateur $\AddList$ est défini dans la macro \texttt{\textbackslash{}AddList}.
}.

\begin{enumerate}
	\item \textit{Ajout d'un nouvel élément à droite.}

	      \verb+\Append**{L}{5}+ donne \Append**{L}{5}.


	\item \textit{Ajout d'un nouvel élément à gauche.}

	      \verb+\Prepend**{L}{5}+ donne \Prepend**{L}{5}.


	\item \textit{Extraction d'un élément -- Version pseudo-automatique.}

	      \verb+\PopAt**{e}{L}{3}+ donne \PopAt**{e}{L}{3} avec un calcul fait automatiquement par la macro. Notez qu'ici on doit indiquer où stoker l'élément extrait.
	      
	      Bien entendu \verb+\PopAt**{e}{L}{1}+ produit \PopAt**{e}{L}{1} sans écrire $\ListUntil{L}{0} \AddList \ListFrom{L}{2}$ puisque pour le package les indices des listes commencent toujours à $1$.
	      
	      Il est autorisé de taper \verb+\PopAt**{e}{L}{k}+ pour obtenir \PopAt**{e}{L}{k}.
	      Par contre, \verb+\PopAt**{e}{L}{k-1}+ aboutit à \PopAt**{e}{L}{k-1} ce qui est très moche ! 
	      Dans ce cas, tapez \verb+\PopAt**{e}{L}{k-2 | k-1 | k}+ afin d'aider la macro à produire \PopAt**{e}{L}{k-2 | k-1 | k}.
\end{enumerate}



\subsubsection{Extraction d'une sous-liste}

\begin{enumerate}
	\item \textit{Extraction d'éléments consécutifs.}

	      Lorsque les calculs automatiques ne sont pas faisables, on devra tout indiquer comme dans \verb+\KeepLR{L}{k - 2}{k}+
	      \footnote{
	      	Le nom de la macro vient de \myquote{keep left and right} soit \myquote{garder à droite et à gauche}.
		  }
		  afin d'avoir \KeepLR{L}{k - 2}{k}.
		  
	\item \textit{Extractions juste à droite, ou juste à gauche.}

	      \verb+\KeepL{L}{k}+ permet d'afficher \KeepL{L}{k} et \verb+\KeepR{L}{k}+ permet quant à lui d'écrire \KeepR{L}{k}
	      \footnote{
	      	Les noms des macros viennent de \myquote{keep left} et \myquote{keep right} soit \myquote{garder à gauche} et \myquote{garder à droite}.
		  }.
\end{enumerate}



\subsubsection{Parcourir une liste}

Les macros \verb+\ForInList+ et \verb+\ForInListRev+ facilitent la rédaction de boucle sur une liste parcourue de façon déterministe.

\codeasideoutput{examples/algo-additional/forinlist.tex}

\newpage

\codeasideoutput{examples/algo-additional/forinlist-rev.tex}

\end{document}

