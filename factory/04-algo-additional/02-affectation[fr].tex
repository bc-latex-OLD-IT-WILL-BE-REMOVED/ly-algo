\documentclass[12pt,a4paper]{article}

\makeatletter
	\usepackage[utf8]{inputenc}
\usepackage[T1]{fontenc}

\usepackage{dsfont}

\usepackage[french]{babel,varioref}

\usepackage[top=2cm, bottom=2cm, left=1.5cm, right=1.5cm]{geometry}
\usepackage{enumitem}

\usepackage{color}
\usepackage{hyperref}
\hypersetup{
    colorlinks,
    citecolor=black,
    filecolor=black,
    linkcolor=black,
    urlcolor=black
}

\usepackage{ifplatform}
\usepackage{import}

\usepackage{multicol}

\usepackage{tcolorbox}

\usepackage{amsthm}

\usepackage{ifplatform}

\usepackage{cbdevtool}
\usepackage{lymath}

% MISC

\setlength{\parindent}{0cm}
\setlist{noitemsep}


\theoremstyle{definition}
\newtheorem*{remark}{Remarque}


\usepackage[raggedright]{titlesec}

\titleformat{\paragraph}[hang]{\normalfont\normalsize\bfseries}{\theparagraph}{1em}{}
\titlespacing*{\paragraph}{0pt}{3.25ex plus 1ex minus .2ex}{0.5em}

\setlength{\parindent}{0cm}

\newenvironment{frame-gene}[1][]{
	\begin{tcolorbox}[
		title        = #1, 
		colbacktitle = black!10!white, 
		colback      = white, 
		coltitle     = black,
		fonttitle    = \bfseries\itshape\small, 
		breakable,
		center title]
}{
	\end{tcolorbox}
}

\newcommand\myquote[1]{{\itshape \og #1 \fg}}

\newcommand\squaremacro{$x^2$}

	% == PACKAGES USED == %

\usepackage{tcolorbox}
\usepackage{xparse}


% == DEFINITIONS == %

\tcbuselibrary{breakable}


% Source used : 
%	* https://tex.stackexchange.com/a/26873/6880

\ExplSyntaxOn

\NewDocumentCommand{\instringTF}{mmmm}
 {
  \oleks_instring:nnnn { #1 } { #2 } { #3 } { #4 }
 }

\tl_new:N \l__oleks_instring_test_tl

\cs_new_protected:Nn \oleks_instring:nnnn
 {
  \tl_set:Nn \l__oleks_instring_test_tl { #1 }
  \regex_match:nnTF { \u{l__oleks_instring_test_tl} } { #2 } { #3 } { #4 }
 }

\ExplSyntaxOff

	% == PACKAGES USED == %

\usepackage[french, vlined]{algorithm2e}


% == DEFINITIONS == %

% Algo - Frames

\newenvironment{algo*}
	{\begin{algorithm}[H]}
	{\end{algorithm}}


\newenvironment{algo}[1][1]{
	\centering
	\begin{tcolorbox}[
		colback = white,
		width=#1\linewidth,
		breakable
	]
	\begin{algo*}
}{
	\end{algo*}
	\vspace{-0.5em}
	\end{tcolorbox}
}


\newcommand\addalgoblank[1][]{
   \ifthenelse{ \equal{#1}{} }
      	{\vspace{.2em}}
      	{\foreach \n in {0,...,#1}{\vspace{.2em}}}
}
	
	\usepackage{02-affectation}
\makeatother


\begin{document}

%\section{Des outils pour les algorithmes} \label{algo-extra}

\subsection{Affectations simples ou multiples}

\subsubsection{Affectation simple avec une flèche}

Les affectations simples classiques $x \Store 3$ et $3 \PutIn x$ se tapent \verb+$x \Store 3$+ et \verb+$3 \PutIn x$+ respectivement où les macros \verb+\Store+ et \verb+\PutIn+ sont des opérateurs mathématiques.


\subsubsection{Affectation simple avec un signe égal décoré}

On peut aussi préférer la notation $x \Store* 3$ ou $x \Store** 3$. Ceci se tape via \verb+$x \Store* 3$+ et \verb+$x \Store** 3$+ respectivement.


\subsubsection{Affectation multiple}

Pour finir, les macros \verb+\MStore+ et \verb+\MPutIn+ servent pour les affectations multiples en parallèle comme dans $a, b, c \MStore x, y, z$ ou $x, y, z \MPutIn a, b, c$ pour indiquer que les dernières valeurs de $x$, $y$ et $z$ sont affectées aux variables $a$, $b$ et $c$.

\medskip

{\itshape \textbf{ATTENTION !} La multi-affectation se faisant en parallèle, le résultat de $a, b, c \MStore 2, a + b, c - b$ ne sera pas semblable à celui de $a \Store 2$ suivi de $b \Store a + b$ puis de $c \Store c - b$ car dans le second cas les variables $a$ et $b$ évoluent avant de nouvelles affectations simples. En fait la multi-affectation précédente correspond aux actions suivantes.}

\medskip

\begin{algo}
	\caption{Comment $a, b, c \MStore 2, a + b, c - b$ fonctionne-t-il ?}

	$a_{memo} \Store a$ ;
	$b_{memo} \Store b$ ;
	$c_{memo} \Store c$
	\\
	\addalgoblank
	$a \Store 2$
	\\
	$b \Store a_{memo} + b_{memo}$
	\\
	$c \Store c_{memo} - b_{memo}$
\end{algo}


\end{document}

