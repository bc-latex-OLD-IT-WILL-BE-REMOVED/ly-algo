\documentclass[12pt,a4paper]{article}

\makeatletter
    \usepackage[utf8]{inputenc}
\usepackage[T1]{fontenc}

\usepackage{dsfont}

\usepackage[french]{babel,varioref}

\usepackage[top=2cm, bottom=2cm, left=1.5cm, right=1.5cm]{geometry}
\usepackage{enumitem}

\usepackage{color}
\usepackage{hyperref}
\hypersetup{
    colorlinks,
    citecolor=black,
    filecolor=black,
    linkcolor=black,
    urlcolor=black
}

\usepackage{ifplatform}
\usepackage{import}

\usepackage{multicol}

\usepackage{tcolorbox}

\usepackage{amsthm}

\usepackage{ifplatform}

\usepackage{cbdevtool}
\usepackage{lymath}

% MISC

\setlength{\parindent}{0cm}
\setlist{noitemsep}


\theoremstyle{definition}
\newtheorem*{remark}{Remarque}


\usepackage[raggedright]{titlesec}

\titleformat{\paragraph}[hang]{\normalfont\normalsize\bfseries}{\theparagraph}{1em}{}
\titlespacing*{\paragraph}{0pt}{3.25ex plus 1ex minus .2ex}{0.5em}

\setlength{\parindent}{0cm}

\newenvironment{frame-gene}[1][]{
	\begin{tcolorbox}[
		title        = #1, 
		colbacktitle = black!10!white, 
		colback      = white, 
		coltitle     = black,
		fonttitle    = \bfseries\itshape\small, 
		breakable,
		center title]
}{
	\end{tcolorbox}
}

\newcommand\myquote[1]{{\itshape \og #1 \fg}}

\newcommand\squaremacro{$x^2$}

\makeatother


\begin{document}

\newpage
\section{Historique}

Nous ne donnons ici qu'un très bref historique de \verb+lyalgo+ côté utilisateur principalement.
Tous les changements sont disponibles uniquement en anglais dans le dossier \verb+change-log+ : voir le code source de \verb+lyalgo+ sur \verb+github+.

\begin{description}[leftmargin=1em]
    \setlength\itemsep{1em}


% --------------- %

%    \item[2019-12-28] Nouvelle version mineure \verb+0.2.0-beta+.
%    \begin{itemize}
%        \item La macro ``\PopAt**`` a cahnég de comportement has changed. %ulti-affectation is used so as to have a coherent behavior.
%
%
%
%
%        **Additional macros:** the macro ``ForRange**`` as an easy way to have a symbolic notation of a ``FOR`` loop.
%
%
%        **¨Doc:** all the ¨latex examples are now organized like the ¨tikz ones.
%
%
%        ==========
%        2019-10-22
%        ==========
%
%        **"Algorithmic flowcharts":** ``to[zigzag]``, ``to[acbackloopleft]`` and ``to[acbackloopright]`` simplify a lot the drawing of "orthopolylines" specific to algorithmic flowcharts.
%    \end{itemize}


% --------------- %

    \item[2019-10-21] Nouvelle version sous mineure \verb+0.1.1-beta+.
    \begin{itemize}
        \item Des nouvelles macros pour les affectations.
        \begin{itemize}
        	\item \verb+\MStore+ et \verb+\MPutIn+ servent à rédiger des affectations multiples en parallèle.

        	\item \verb+\Store*+ et \verb+\Store**+ produisent des écritures symboliques de l'affectation simple via des signes $=$ décorés.
        \end{itemize}

        \item \verb+\CSinterval+ sert à rédiger des intervalles à la sauce informatique.
    \end{itemize}


% --------------- %

    \item[2019-10-19] Nouvelle version mineure \verb+0.1.0-beta+.
    \begin{itemize}
        \item Ajout d'outils pour faciliter le dessin d'ordinogrammes via \verb+TiKz+.
    \end{itemize}


% --------------- %

    \item[2019-10-18] Le documentation a enfin son journal des changements principaux.


% --------------- %

    \item[2019-09-03] Première version \verb+0.0.0-beta+ du package.
\end{description}

\end{document}
